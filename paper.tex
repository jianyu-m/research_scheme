
\documentclass{article}

\usepackage{times}
\usepackage{fullpage}
% \usepackage{mintåed}
\usepackage{cite}
\usepackage{graphicx}
\usepackage{hyperref}
\usepackage{bbding}
\usepackage{pifont}
\usepackage{wasysym}
\usepackage{amssymb}
\usepackage{tightenum}
\usepackage{xspace}


\title{New Systems for Preserving Big-data Privacy in Clouds}
% \date{\today}
\author{Jianyu Jiang}

\let\vv\texttt

%% Title stuff.
%\newcommand{\mytitle}[0]{\textbf {Stable Multithreading: \\
%A New Paradigm for Reliable and Secure Threads}}
%\newcommand{\mykeywords}[0]{Deterministic Multithreading, Stable 
%Multithreading, Reliability, Security, Software Model Checking,
%State Space Reduction, State Machine Replication}




%% Terminology of our own internal techniques.
\newcommand{\kakute}[0]{\textsc{Kakute}\xspace}
\newcommand{\confluence}[0]{\textsc{Confluence}\xspace}
\newcommand{\pig}[0]{\textsc{Pig}\xspace}
\newcommand{\hadoop}[0]{\textsc{Hadoop}\xspace}
\newcommand{\maat}[0]{\textsc{Uranus}\xspace}
\newcommand{\chref}[1]{\S\ref{#1}}
\newcommand{\lazyp}[0]{Reference Propagation\xspace}
\newcommand{\tagcache}[0]{Tag Sharing\xspace}
\newcommand{\func}[1]{\textsc{#1}}


% Objective 1
\newcommand{\appsn}[0]{4\space}
\newcommand{\appeval}[0]{six\xspace}
\newcommand{\overheadcomp}[0]{50\%\xspace}
\newcommand{\overheadmem}[0]{1x\xspace}
\newcommand{\dagfull}[0]{Directed acyclic graph\space}
\newcommand{\timelow}[0]{60\%\space}
\newcommand{\timehigh}[0]{4.5X\space}
\newcommand{\timeavg}[0]{32.3\%\space}


%% Systems and techniques names.
% \newcommand{\xxx}[0]{\textsc{Gaia}\xspace}
\newcommand{\paxos}[0]{\textsc{Paxos}\xspace}
\newcommand{\spaxos}[0]{S-Paxos\xspace}
\newcommand{\zookeeper}{ZooKeeper\xspace}
\newcommand{\libpaxos}{libPaxos\xspace}
\newcommand{\dare}{DARE\xspace}
\newcommand{\crane}{\textsc{Crane}\xspace}
\newcommand{\falcon}{\textsc{Apus}\xspace}
\newcommand{\tripod}{\textsc{Tripod}\xspace}
\newcommand{\mesos}{\textsc{Mesos}\xspace}

\newcommand{\dmt}[0]{DMT\xspace}
\newcommand{\smt}[0]{StableMT\xspace}
\newcommand{\smr}[0]{SMR\xspace}
\newcommand{\racepro}[0]{\textsc{RacePro}\xspace}
\newcommand{\tern}[0]{\textsc{Tern}\xspace}
\newcommand{\peregrine}[0]{\textsc{Peregrine}\xspace}
\newcommand{\parrot}[0]{\textsc{Parrot}\xspace}
\newcommand{\grace}[0]{Grace\xspace}
\newcommand{\coredet}[0]{\textsc{CoreDet}\xspace}
\newcommand{\kendo}[0]{Kendo\xspace}
\newcommand{\dthreads}[0]{\textsc{DThreads}\xspace}
\newcommand{\determinator}[0]{Determinator\xspace}
\newcommand{\dbug}[0]{\textsc{dbug}\xspace}
\newcommand{\ecosys}[0]{\parrot-\dbug}
\newcommand{\ldpreload}[0]{LD\_PRELOAD\xspace}

\newcommand{\mutexlock}[0]{\texttt{pthread\_mutex\_lock}\xspace}
\newcommand{\send}[0]{\texttt{send}\xspace}
\newcommand{\accept}[0]{\texttt{accept}\xspace}
\newcommand{\connect}[0]{\texttt{connect}\xspace}
\newcommand{\recv}[0]{\texttt{recv}\xspace}
\newcommand{\sockread}[0]{\texttt{read}\xspace}
\newcommand{\close}[0]{\texttt{close}\xspace}
\newcommand{\select}[0]{\texttt{select}\xspace}
\newcommand{\poll}[0]{\texttt{poll}\xspace}
\newcommand{\epoll}[0]{\texttt{epoll}\xspace}
\newcommand{\randfunc}[0]{\texttt{rand}\xspace}
\newcommand{\srandfunc}[0]{\texttt{srand}\xspace}
\newcommand{\gettimeofday}[0]{\texttt{gettimeofday}\xspace}
\newcommand{\us}[0]{\(\mu\text{s}\)\xspace}
%\newcommand{\N}[0]{$N$\xspace}
%\newcommand{\M}[0]{$M$\xspace}
%\newcommand{\C}[0]{$C$\xspace}
%\newcommand{\timev}[0]{\texttt{T_{v}}\xspace}

%% Application names.
\newcommand{\azure}{Microsoft Azure\xspace}
\newcommand{\libsafe}{Libsafe\xspace}
\newcommand{\apache}{Apache\xspace}
\newcommand{\clamav}{ClamAV\xspace}
\newcommand{\ab}{ApacheBench\xspace}
\newcommand{\mysql}{MySQL\xspace}
\newcommand{\sysbench}{SysBench\xspace}
\newcommand{\mplayer}[0]{{MPlayer}\xspace}
\newcommand{\mencoder}[0]{\v{mencoder}\xspace}
\newcommand{\pbzip}[0]{\v{PBZip2}\xspace}
\newcommand{\aget}[0]{\v{aget}\xspace}
\newcommand{\mongoose}[0]{\v{Mongoose}\xspace}
\newcommand{\pfscan}[0]{\v{pfscan}\xspace}
\newcommand{\fft}[0]{\v{fft}\xspace}
\newcommand{\luc}[0]{\v{lu\_cb}\xspace}
\newcommand{\lun}[0]{\v{lu\_ncb}\xspace}
\newcommand{\barnes}[0]{\v{barnes}\xspace}
\newcommand{\radix}[0]{\v{radix}\xspace}
\newcommand{\radiosity}[0]{\v{radiosity}\xspace}
\newcommand{\waters}[0]{\v{water-spatial}\xspace}
\newcommand{\watern}[0]{\v{water-nsquared}\xspace}
\newcommand{\oceanncp}[0]{\v{ocean}\xspace}
\newcommand{\oceancp}[0]{\v{ocean}\xspace}
\newcommand{\ocean}[0]{\v{ocean}\xspace}
\newcommand{\fmm}[0]{\v{fmm}\xspace}
\newcommand{\volrend}[0]{\v{volrend}\xspace}
\newcommand{\cholesky}[0]{\v{cholesky}\xspace}
\newcommand{\streamcluster}[0]{\v{streamcluster}\xspace}
\newcommand{\blackscholes}[0]{\v{blackscholes}\xspace}
\newcommand{\swaptions}[0]{\v{swaptions}\xspace}
\newcommand{\bodytrack}[0]{\v{bodytrack}\xspace}
\newcommand{\bodytrackopenmp}[0]{\v{bodytrack-openmp}\xspace}
\newcommand{\ferret}[0]{\v{ferret}\xspace}
\newcommand{\dedup}[0]{\v{dedup}\xspace}
\newcommand{\raytrace}[0]{\v{raytrace}\xspace}
\newcommand{\canneal}[0]{\v{canneal}\xspace}
\newcommand{\racey}[0]{\v{racey}\xspace}
\newcommand{\freqmine}[0]{\v{freqmine}\xspace}
\newcommand{\vips}[0]{\v{vips}\xspace}
\newcommand{\xtwosixfour}[0]{\v{x264}\xspace}
\newcommand{\fluidanimate}[0]{\v{fluidanimate}\xspace}
\newcommand{\facesim}[0]{\v{facesim}\xspace}
\newcommand{\rtviewraytrace}[0]{\v{rtview\_raytrace}\xspace}
\newcommand{\wordcount}[0]{\v{word\_count}\xspace}
\newcommand{\klee}[0]{\textsc{klee}\xspace}
\newcommand{\woodpecker}[0]{\textsc{woodpecker}\xspace}
\newcommand{\splashx}[0]{\mbox{SPLASH-2x}\xspace}
\newcommand{\splash}[0]{\mbox{SPLASH-2}\xspace}
\newcommand{\parsec}[0]{\mbox{PARSEC}\xspace}
\newcommand{\phoenix}[0]{\mbox{Phoenix}\xspace}
\newcommand{\pthread}[0]{\mbox{Pthreads}\xspace}

%% Application names (continue).
\newcommand{\kmeans}[0]{\v{kmeans}\xspace}
\newcommand{\kmeanspthread}[0]{\v{kmeans-pthread}\xspace}
\newcommand{\linearregre}[0]{\v{linear-regression}\xspace}
\newcommand{\linearregrepthread}[0]{\v{linear-regression-pthread}\xspace}
\newcommand{\matrixmult}[0]{\v{matrix-multiply}\xspace}
\newcommand{\matrixmultpthread}[0]{\v{matrix-multiply-pthread}\xspace}
\newcommand{\wordcnt}[0]{\v{word-count}\xspace}
\newcommand{\wordcntpthread}[0]{\v{word-count-pthread}\xspace}
\newcommand{\stringmatch}[0]{\v{string-match}\xspace}
\newcommand{\stringmatchpthread}[0]{\v{string-match-pthread}\xspace}
\newcommand{\histogram}[0]{\v{histogram}\xspace}
\newcommand{\histogrampthread}[0]{\v{histogram-pthread}\xspace}
\newcommand{\pca}[0]{\v{pca}\xspace}
\newcommand{\pcapthread}[0]{\v{pca-pthread}\xspace}

%% Application names (continue).
\newcommand{\partition}[0]{\mbox{\v{partition}}\xspace}
\newcommand{\nthelement}[0]{\mbox{\v{nth\_element}}\xspace}
\newcommand{\partialsort}[0]{\mbox{\v{partial\_sort}}\xspace}
\newcommand{\ua}[0]{\v{ua}\xspace}
\newcommand{\is}[0]{\v{is}\xspace}
\newcommand{\npb}[0]{\mbox{NPB}\xspace}
\newcommand{\imagick}[0]{\mbox{ImageMagick}\xspace}
\newcommand{\openldap}[0]{{OpenLDAP}\xspace}
\newcommand{\redis}[0]{{Redis}\xspace}
\newcommand{\memcached}[0]{{Memcached}\xspace}
\newcommand{\bdb}[0]{{Berkeley DB}\xspace}
\newcommand{\openmp}[0]{{OpenMP}\xspace}
\newcommand{\libgomp}[0]{\v{libgomp}\xspace}
\newcommand{\vtune}[0]{\v{VTune}\xspace}

%% concurrency attack study stats.
\newcommand{\noldattacks}[0]{46\xspace}
\newcommand{\nattacks}[0]{23\xspace}

%% imagick programs
\newcommand{\convertshear}[0]{\v{convert\_shear}\xspace}
\newcommand{\montage}[0]{\v{montage}\xspace}

\newcommand{\nprog}[0]{108\xspace}
\newcommand{\nrealprog}[0]{55\xspace}
\newcommand{\nstl}[0]{33\xspace}
\newcommand{\nimagick}[0]{14\xspace}
\newcommand{\nparsec}[0]{15\xspace}
\newcommand{\nphoenix}[0]{14\xspace}
\newcommand{\nsplash}[0]{14\xspace}
\newcommand{\nnpb}[0]{10\xspace}
\newcommand{\nbenchmarks}[0]{53\xspace}
\newcommand{\ndatapartition}[0]{86\xspace}
\newcommand{\nprogadhocsync}[0]{5\xspace}
\newcommand{\nprogtimeout}[0]{5\xspace}

\newcommand{\nprognohints}[0]{18\xspace}
\newcommand{\nprogneedhints}[0]{90\xspace}
\newcommand{\nprognondethints}[0]{9\xspace}
\newcommand{\nprognondetandnetwork}[0]{11\xspace} % the UA program is excluded.
\newcommand{\nprognonondethints}[0]{99\xspace}
\newcommand{\nproglineuphints}[0]{81\xspace}
\newcommand{\nproggenericlineuphints}[0]{43\xspace}
\newcommand{\nprogspecificlineuphints}[0]{38\xspace}
\newcommand{\nlineofhints}[0]{109\xspace}
\newcommand{\nlineofcomputehints}[0]{87\xspace}
\newcommand{\nlineofnondethints}[0]{22\xspace}
\newcommand{\hintsperprog}[0]{1.2\xspace}
\newcommand{\nlineupfails}[0]{12\xspace}

%% effects of perfomance hints.
\newcommand{\genericnolineup}[0]{500\%\xspace}
\newcommand{\genericlineup}[0]{0.8\%\xspace}
\newcommand{\specificnolineup}[0]{460\%\xspace}
\newcommand{\specificlineup}[0]{19.1\%\xspace}
\newcommand{\nondetnohints}[0]{830\%\xspace}
\newcommand{\nondethints}[0]{42.1\%\xspace}
\newcommand{\overallnohints}[0]{510\%\xspace}
\newcommand{\overallhints}[0]{11.9\%\xspace}

%% our geometric mean overhead on standard workload.
\newcommand{\meanoverhead}[0]{12.7\%\xspace}
\newcommand{\meanrealoverhead}[0]{6.9\%\xspace}
\newcommand{\meanbenchoverhead}[0]{19.0\%\xspace}

%% model checking reduction.
%% nprogshrink, totally 56: 50 programs with out nondet hints, and 6 programs with network or nondet hints.
\newcommand{\shrinkscale}[0]{$10^{6}$--$10^{19734}$\xspace}
\newcommand{\nprogshrink}[0]{56\xspace}
\newcommand{\nprognondetshrink}[0]{5\xspace}
\newcommand{\nprogverifiedxxx}[0]{99\xspace}
\newcommand{\nprogverifieddbug}[0]{43\xspace}

%% overheads in comparison table.
\newcommand{\nprogcompared}[0]{25\xspace}
\newcommand{\xxxcompoverhead}[0]{11.8\%\xspace}
\newcommand{\dthreadssyncoverhead}[0]{150.0\%\xspace}
\newcommand{\dthreadssyncoverheadnoflui}[0]{112.5\%\xspace}
%\newcommand{\dthreadsoverhead}[0]{1,173\%\xspace}
\newcommand{\dthreadsexampleoverhead}[0]{7.7$\times$\xspace}
\newcommand{\coredetoverhead}[0]{115.1\%\xspace}
\newcommand{\overeach}[0]{10$\times$\xspace}
\newcommand{\overcombined}[0]{4$\times$\xspace}

% Stats
\newcommand{\fasterDARElow}[0]{7.9\%\xspace}
\newcommand{\fasterDARE}[0]{3.3X\xspace}
\newcommand{\xxxlatencythree}[0]{8.2\xspace}
\newcommand{\xxxlatencyonezerofive}[0]{31.6\xspace}

\newcommand{\comptradlow}[0]{32.3X\xspace} % TBD
\newcommand{\comptradhigh}[0]{85.8X\xspace} % TBD

\newcommand{\tradlatencyincreaselow}[0]{30.3\%\xspace} % TBD
\newcommand{\tradlatencyincreasehigh}[0]{156.8\%\xspace} % TBD
\newcommand{\systemcostlow}[0]{36.5\%\xspace} % TBD
\newcommand{\systemcosthigh}[0]{63.7\%\xspace} % TBD
\newcommand{\xxxscalability}[0]{3.8x\xspace} % TBD
\newcommand{\darescalability}[0]{11.7x\xspace} % TBD

% Tripod result.
\newcommand{\tputoverhead}[0]{3.22\%\xspace}
\newcommand{\latencyoverhead}[0]{3.31\%\xspace}

\newcommand{\eg}{{e.g.}}
\newcommand{\ie}{{i.e.}}
\newcommand{\etc}{{etc}}
\newcommand{\para}[1]{\vspace{.00in}\noindent{\bf #1}}
\newcommand{\wrt}{{w.r.t. }}
\newcommand{\cf}{{cf. }}
\newcommand{\vs}{{vs.}\xspace}

% Architecture of this proposal
% the level of verification
% 1. verification of SGX
% 2. verification of runtime (uranus)
% 3. verification of program

% the verification target
% 1. functional correctness (spec <- state machine spec)
% 2. security properties: 
%    1) type-safety
%    2) confidentiality
%    3) integrity
%    4) rop (memory safety)

%    5) integrity
%    6) confidentiality
%    7) secure measurement

%    8) termination?
%    9) error-crash?

\begin{document}
\maketitle

\section{Introduction}
% secured-critical systems are important
In this big-data era, many business vendors (e.g., Uber) store data on clouds.
Meanwhile, business vendors and third-parties implement self-defined queries
(e.g., MapReduce) to process data, causing two severe privacy problems. First,
third-parties can easily leak sensitive fields in data records (e.g., credit
cards in Uber orders) through query results. Second, careless or malicious cloud
providers (e.g., Amazon) can observe the data being queried.

First, to confine malicious third-parties, we has built Kakute, the first Data
Flow Tracking (DFT) system for big-data. Kakute provides easy-to-use APIs for
business vendors to tag sensitive data fields, it then automatically tracks
unmodified queries and prevents sensitive data flowing to query results. A
challenge in existing DFT systems is that propagating tags in data-intensive
computations is too slow (e.g., a notable DFT system incurs 128X performance
overhead compared to native, insecure queries). By leveraging subtle efficiency
features of big-data queries, we have created two fast tag propagation
techniques. Our Kakute prototype~\cite{kakute:acsac17} incurs
merely 32.3\% performance overhead.

Second, to confine malicious cloud providers, we will leverage the Intel SGX
(Software Guard Extensions) hardware to build the first privacy-preserving
compiler for unmodified big-data queries. Existing SGX-based systems for
big-data have two major challenges: they have to rewrite the queries from Java
to SGX-compatible C++, or their trusted components running in SGX are too large
(e.g., an entire JVM). To tackle these challenges, our compiler runs only the
Java big-data queries in SGX using a thin, verified just-in-time translator
created by us. The compiler also carries our new efficient SGX-based runtime
techniques.

The success of my research will effectively preserve big-data privacy in
clouds and benefit most people.


% approach of building secured-critical system

% trusted hardware
\section{Objective}
\subsection{Data Flow Tracking (DFT) system for private clouds (Finished Work)} 
We will create Kakute, a fast DFT system that can track and prevent sensitive data
leakage in self-defined big-data queries. We will make Kakute support diverse
big-data queries on large, popular datasets, and we will make Kakute incur
reasonable performance overhead compared to native, insecure queries.

\subsection{Compiler for big-data privacy in public clouds (Future Work)} 
Our compiler will support unmodified big-data frameworks by creating a thin,
verified translator that automatically translates Java bytecode to
SGX-compatible code. We will quantitatively evaluate the performance overhead of
our compiler and whether it can protect data privacy against real-world
privileged attacks.

\section{Related Work}
\subsection{Big-data computing frameworks} 
\label{sec:bigdata}\vspace{-.075in}

Big-data frameworks (\eg, Spark~\cite{nsdi12:spark} and 
MapReduce~\cite{hadoop}) are popular for computations on tremendous amounts 
of data records. These frameworks provide self-defined Java functions (\eg, 
\vv{map}/\vv{reduce}) to let computation providers write their algorithms, 
and the frameworks automatically apply these functions on the data stored 
across computers in parallel.

To avoid excessive computation, big-data frameworks adopt a \emph{lazy 
transformation} approach~\cite{pig:vldb08,nsdi12:spark}. Spark 
often uses lazy transformations (\eg, \vv{map}), and calls to 
these transformations only create a data structure called \func{RDD} with 
\emph{lineage} (the sequence of transformations on data records).
Actual transformations are only triggered when materialization 
operations (\eg, \vv{collect}/\vv{count}) are called. Collecting 
operations trigger transformations only along lineages, so unnecessary 
computations are avoided. \textbf{Objective 1} will leverage lazy 
transformation to create a fast data flow tracking technique: \lazyp 
(\S\ref{sec:obj1}).

\subsection{Software-based privacy techniques}
\label{sec:dft}\vspace{-.075in}

Data Flow Tracking (DFT) is a powerful mandatory access control technique for 
preventing sensitive data leakage~\cite{taintdroid:osdi10}. DFT attaches 
a tag to a variable (or object), and this tag will propagate during 
computations on the variable at runtime. DFT is used in various areas, 
including preventing sensitive data (\eg contacts) leakage in smart phones 
(TaintDroid~\cite{taintdroid:osdi10}), web services~\cite{cloudfence:raid13}, 
and server programs~\cite{libdft:vee12}. No DFT system exists for big-data 
computing, so \textbf{Objective 1} (\chref{sec:obj1}) will create the 
first DFT system for big-data.

\subsection{Hardware-based privacy techniques}
\label{sec:sgx}\vspace{-.075in}

Trusted Execution Environment (TEE) is a promising technique for
protecting computation in a public cloud even if the cloud's operating systems 
and hypervisors compromise. For instance, Intel-SGX~\cite{intel-sgx}, a 
popular commercial TEE product, runs a program in a hardware-protected 
\emph{enclave}, so code and data are protected from outside. Compared with the 
Homomophic Encryption approach of computing on encrypted data, TEE is much
faster. For instance, a SGX-based system Opaque~\cite{opaque:nsdi17} 
incurs a moderate performance overhead of 30\% compared to native, insecure 
big-data queries.

However, to practically run Java big-data queries with SGX, there are still two
major challenges. First, existing SGX-based systems~\cite{opaque:nsdi17} require
computation providers to manually rewrite the readily pervasive Java queries
into SGX-compatible C++, a time-consuming and error-prone process. Second,
existing SGX-based systems for big-data have too large Trusted Computing Base
(TCB). Existing systems (\eg, SGX-BigMatrix~\cite{bigmatrix:ccs17}) run a whole
language interpreter (\eg, JVM and Python runtime) in enclaves, causing a too
large (and too dangerous) TCB: JVM code comes from many different
parties/vendors and extremely hard to verified. \textbf{Objective 2}
(\chref{sec:obj2}) tackles these two challenges by building a new just-in-time
compiler.

\section{Research Methodology}
\subsection{Objective 1: 
preventing big-data computation leakage with \kakute}\label{sec:obj1}
\vspace{-.075in}

Although DFT is a powerful access control technique, existing DFT systems incur 
high performance overhead, especially for data-intensive computations. For 
instance, we ran a recent DFT system Phosphor~\cite{oo14:phosphor} in Spark 
with a WordCount algorithm on a small dataset of merely 200MB, and we observed 
128X longer computation time compared with the native Spark 
execution~\cite{kakute:acsac17}. The second challenge is completeness: 
big-data frameworks usually contain \emph{shuffle} operations, which 
redistribute data and results across computers. However, most existing DFT 
systems ignore data flows across computers. For the 
few~\cite{cloudfence:raid13} who support cross-host data flows, transferring 
all tags in shuffles consumes excessive network bandwidth.

\subsubsection{\kakute: a fast, precise DFT system for big-data} 
\label{sec:kakute}\vspace{-.075in}

\kakute{} is the first precise and complete DFT system for big-data 
frameworks. The key insight to address the DFT performance challenge is that 
multiple fields of a record often have the same tags with the same sensitivity 
level. For example, in a Taobao order record $\langle$\vv{time}, 
\vv{userId}, \vv{productID}$\rangle$, only the \vv{userId} field is 
sensitive, while the other fields are insensitive and they can share the same 
tag. Leveraging this insight, we present two new techniques, \lazyp and 
\tagcache. \lazyp avoids unnecessary tag combinations by only keeping the 
\textit{lineage of tags} in the same self-defined queries, while \tagcache 
reduces memory usage by sharing tags among multiple fields in each record. To 
tackle the completeness challenge, \kakute completely captures inter-computer 
data flows (shuffles), and it efficiently reduces the amount of 
transferred DFT tags using \tagcache. Both techniques are illustrated in 
Appendix (c) of this proposal.

Figure~\ref{fig:falcon-protocol} shows \kakute's design. The 
InputTainter component provides easy-to-use APIs for data providers to 
automatically tag sensitive fields. The DFT component is enabled in 
self-defined functions. The Local- and Global-Checker detect and 
prevent illegal flows of sensitive fields (\eg, credit cards flow to IO 
functions in self-defined functions). Shuffle operations across computers are 
intercepted and tags are added. Therefore, DFT is completely captured across 
computers.

We have implemented \kakute and integrated it with Spark. We leverage 
Phosphor~\cite{oo14:phosphor}, an efficient DFT system working in the Java 
byte-code level. \kakute instruments computations of a Spark worker process
to capture data flows inside self-defined-functions. \kakute provides different 
granularities of tracking with two types of tags: \func{Integer}
and \func{Object} tags. \func{Integer} provides 32 distinct tags for 
identifying 32 sensitivity levels, suitable for detecting data leakage and 
performance bugs. \func{Object} provides an arbitrary number of tags, which is 
suitable for data provenance and programming debugging.
% \kakute provides a 
% unified API to tackle diverse problems.
% Based on this unified API, we 
% will implement 4 built-in checkers for \appsn security and reliability 
% problems: sensitive information leakage, data provenance, programming and 
% performance bugs. 

\kakute (\textbf{Objective 1}) strictly prevents
sensitive data flowing to IO functions or query results, but in some scenarios 
it is still desirable to let computation providers acquire aggregation 
results (\eg, the sum of citizens who have got cancer in a country) 
on sensitive fields as long as individual information is not leaked. 
Differential privacy~\cite{Dwork2006Differential} can enforce statistical 
bounds on 
aggregation results and prevent individual information leakage, so it is 
complementary to DFT and has attracted much attention recently.

% Differential privacy is a robust privacy standard proposed by Dwork et
% al.
% His work shows that differential privacy can be achieved by adding
% Laplace distributed noise whose scale is related to the sensitivity of queries.
% This simple algorithm can handle all kinds of queries but inevitably has some
% drawbacks. For instance, the number of query is limited and too much noise will
% be injected to continuous queries. To reduce noise magnitude, a 
% sample-and-aggregation
% framework~\cite{differentialdp:stoc11} is proposed.

\begin{figure}[h]
    \centering
    \begin{minipage}{0.40\textwidth}
        \vspace{-.1in}
        \includegraphics[width=0.95\textwidth]{figures/kakute_arch.pdf}
         \vspace{-.4in}
        \caption{\kakute architecture. \kakute's key components are shaded 
(and in blue).}
        \label{fig:falcon-protocol}
   \end{minipage}
    \hspace{.1in}
    \centering
    \begin{minipage}{0.48\textwidth}
        \vspace{-.1in}
       \includegraphics[width=0.95\textwidth]{figures/uranus_arch.pdf}
        \vspace{-.1in}
       \caption{\maat Uranus's Architecture.}
       \label{fig:maat-arch}
    \end{minipage}
\end{figure}

\subsection{Objective 2: creating a privacy-preserving compiler 
for big-data queries in public clouds}\label{sec:obj2}\vspace{-.075in}

% In order to reduce maintainance costs, more and more sensitive data are stored 
% and processed in public clouds (\eg, Amazon EC2 and Dropbox). Although data can 
% be securely encrypted while in rest, recent real-world privacy breaches have 
% shown that data are often leaked while being processed. The reason is that 
% homomophic encryption is still pre-mature (a notable analytic database 
% CryptoDB~\cite{cryptdb:sosp11} reports a 1000X overhead compared with
% the unencrypted execution), and most data is being processed in plaintext. 

Recent real-world privacy breaches have shown that sensitive data are often 
leaked while being processed in public clouds, including clouds 
compromises on external attacks~\cite{icloud-breach} and 
insider attacks~\cite{top-threats}. Trusted Execution Environment 
(TEE) is a promising technique to protect computation on public clouds even if 
the cloud's operating system is compromised. For example, 
Intel-SGX~\cite{intel-sgx} runs programs in an enclave, so code and data are 
protected and cannot be seen by the attackers. Meanwhile, SGX is good fit for 
big-data queries because these queries are data-intensive in userspace and they 
hardly invoke system calls (OS kernel can easily break SGX's security on 
memory). A latest big-data analytic system Opaque~\cite{opaque:nsdi17} reports 
only 30\% overhead compared to native, insecure executions.

Despite recent SGX-based systems (Opaque~\cite{opaque:nsdi17}, 
VC3~\cite{vc3:sp15}, Azure/Coco~\cite{azure:coco}, and 
SGX-BigMatrix~\cite{bigmatrix:ccs17}) for big-data are promising, two major 
challenges remain. First, these systems require completely rewriting the 
readily pervasive and familiar Java big-data queries 
into C++~\cite{opaque:nsdi17,vc3:sp15,azure:coco}, a time-consuming and 
error-prone process. Second, to easy implementation, existing systems often 
run an entire language runtime (\eg, JVM or Python 
runtime~\cite{bigmatrix:ccs17}) within SGX, causing the Trusted Computing Base 
(TCB) to be too large (\eg, JVM has millions of lines of code 
from many companies and is vulnerable to insider attacks~\cite{top-threats}).
% SGX-BigMatrix~\cite{bigmatrix:ccs17} proposes a secure and oblivious 
% vectorization
% abstraction for Python, but it needs to rewrite programs with its abstraction.

\vspace{-.15in}
\subsubsection{\maat: a privacy-preserving Java interpreter 
for big-data queries} 
\label{sec:ift-problem}\vspace{-.075in}

\maat will be the first interpreter that runs unmodified Java big-data
queries in SGX enclaves securely with minimal TCB (\ie, the TCB contains only
SGX and self-defined code itself). \maat works as a Java JIT interpreter which
automatically executes self-defined big-data functions (\eg,
\vv{map}/\vv{reduce}) into SGX enclaves. 

The goal of \maat is to preserve the confidentiality of data while being 
processed in public clouds. Other attacks such as changing execution paths 
(\ie, integrity) have been well defended in prior work~\cite{jitguard:ccs17}, 
and \maat can directly use it.

In \maat, SGX,
computation providers are trusted, but cloud providers are malicious. Figure 
\ref{fig:maat-arch} shows the architecture of \maat. In \maat, both the 
translation of self-defined code and the execution of the code are protected by 
enclaves, so that even if the cloud's OS compromised, it cannot see the 
executions of big-data queries or inject malicious code into the queries during 
\maat's translation. Translated functions are put in \maat's code cache for 
future reuse.
% Self-defined functions and data processed by these 
% functions are protected by \maat . The Java byte-codes of these functions are 
% compiled to enclave-comptabile assembly code.

Two \maat software components run in SGX: the \maat interpreter and our own
management library. The interpreter is a thin layer which executes a Java
bytecode instructions in enclaves. The memory management library is for our own
use of encryption/decryption on data records and maintaining SGX memory for the
queries at runtime. We will proactively implement these two components to be
easy to verify (use as few as function recursive calls and loops) as in other
verification practice~\cite{xi:sosp17}, and we plan to use state-of-the-art
verification techniques~\cite{xi:sosp17} to verify both of them. Then, we do not
need to include them in \maat's TCB, greatly reducing its TCB.
% Specifically, we plan to implement the two 
% components .


% We don't plan to use OpenJDK as our management library 
% because it contains up to millions lines of code, which is impractical for 
% formal verifications as it is time-consuming with current verification methods 
% (\chref{sec:others-work}).

One subtle performance challenge for \maat is that it should have reasonable 
performance overhead compared to native executions. When calling into and 
out of a function in enclaves, an ECALL and OCALL will be invoked in the CPU 
and enclave transitions are invoked. Such transitions are several hundred 
times slower than user function calls. Moreover, encryption and decryption on 
data records are invoked during such transitions. Our study on a SGX-based 
big-data system Opaque~\cite{opaque:nsdi17} shows that it incurs 3.4k enclave
transitions for processing only 10k data (with two queries \vv{select} and 
\vv{groupBy}), which confirms the challenge.

To mitigate this challenge, we will create a new enclave runtime abstraction 
called Data-locality-aware Asynchronous Enclave calls (DAE). DAE converts the 
synchronous enclave calls (similar to Java function calls) to asynchronous, 
data-locality-aware calls into enclaves. Specifically, DAE will run a number of 
$n$ processes ($E_{1}$ to $E_{n}$) in an enclave on each computer. When a 
JVM process $P$ calls a big-data query function, the call and its parameters 
are appended to a queue to DAE, and DAE arranges a process $E_{i}$ with good 
data locality (\eg, according to prior arrangements and the decrypted data 
held by $E_{i}$) to execute the call. The call result is appended to a return 
queue of the DAE for process $P$. We expect that DAE will achieve reasonable 
performance, data locality, and parallelism.

By realizing a privacy-preserving Java JIT compiler for public clouds, \maat has
broad applications in other security areas. First, we will augment the
interpreter to automatically executes the big-data queries with dangerous access
patterns on particular data into those without (\eg, oblivious
executions~\cite{bigmatrix:ccs17}). Second, we will further enhance DAE to
support well isolated, secured operating system calls (\eg, library operating
system calls~\cite{graphene:atc17}), so that \maat will not only benefit
big-data queries, but other distributed computing paradigms (\eg, graph
queries~\cite{sigmod10:pregel}).


\bibliography{bib/biblio.bib}
\bibliographystyle{abbrv} 

\end{document}